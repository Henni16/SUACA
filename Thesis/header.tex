\usepackage[latin1]{inputenc}
\usepackage{amssymb}
\usepackage{amsthm}
\usepackage[fleqn]{amsmath}
\usepackage[dvipsnames]{xcolor}
\usepackage{relsize}
\usepackage{microtype}
\usepackage{paralist}
\usepackage{graphicx}
\usepackage[super]{nth}
\usepackage{relsize}
\usepackage[linesnumbered, ruled]{algorithm2e}
\usepackage[section]{placeins}
\usepackage{pdfpages}
\usepackage{biblatex}
\addbibresource{cites.bib}



\newcommand{\suaca}{\textbf{SUACA}}
\newcommand{\iaca}{\textbf{IACA}}
\newcommand{\osaca}{\textbf{OSACA}}
\newcommand{\microop}{$\mu$op}
\newcommand{\microops}{$\mu$ops}

%\usepackage{bnf}
%\usepackage{epsf}
\usepackage{appendix}

%\usepackage[a4paper]{geometry}
%\geometry{left=3cm,right=3cm,top=23mm,bottom=25mm,head=14.5pt}

\usepackage{phdthesis}
\setlength{\headheight}{15pt}
\usepackage{listings}
\lstdefinelanguage{myLang}
{
    % list of keywords
    morekeywords={
        movl,
        byte,
        mov,
        cmp,
        jne,
        add,
        jmp,
        else,
        end,
        adc
    },
    sensitive=false, % keywords are not case-sensitive
    morecomment=[l]{//}, % l is for line comment
}
\lstset{showspaces=false,showstringspaces=false,breaklines=true,breakindent=0pt,
        prebreak={},
        postbreak=\mbox{{ }},
    escapeinside={(*@}{@*)},
    language=C,
    basicstyle=\ttfamily,
    keywordstyle=\color{blue}\ttfamily,
    stringstyle=\color{red}\ttfamily,
    commentstyle=\color{green}\ttfamily,
    morecomment=[l][\color{magenta}]{\#}
}
\usepackage{makeidx}          % wir wollen auch einen Index
\usepackage{hyperref}
\renewcommand{\sectionautorefname}{Section}
\renewcommand{\chapterautorefname}{Chapter}
\renewcommand{\algorithmautorefname}{Algorithm}
\usepackage{float}
%\renewcommand*{\figureautorefname}{figure}
\usepackage{tikz}
\usetikzlibrary{matrix,backgrounds, shapes, arrows.meta}
\tikzset{mynode/.style={draw, ellipse, thick, line width = 2pt}}
\usepackage{tkz-graph}
\SetUpEdge[lw         = 1.5pt,
           style={->},
           color      = black,
           labelcolor = white,
           labeltext  = red,
           labelstyle = {draw,text=black}]

\tikzset{bright/.style = {bend right=23, ->}}
\tikzset{bleft/.style = {bend left=23, ->}}

\definecolor{light-gray}{gray}{0.95}
\usepackage{mdframed}
\usepackage{pgfplots}
\usepackage{changepage}
\usepackage{fancyvrb}
\usepackage{wrapfig}
\usepackage{lipsum}
\parindent0pt
\newtheorem{theorem}{Theorem}
\newtheorem{definition}{Definition}




\newenvironment{example}{\VerbatimEnvironment\begin{mdframed}[backgroundcolor=light-gray, roundcorner=10pt,leftmargin=1, rightmargin=1, innerleftmargin=15, innertopmargin=15,innerbottommargin=15, outerlinewidth=1, linecolor=light-gray]
        \begin{center}\begin{BVerbatim}[fontsize=\tiny]}{\end{BVerbatim}\end{center}
    \end{mdframed}}

\newenvironment{Example}{\VerbatimEnvironment\begin{mdframed}[backgroundcolor=light-gray, roundcorner=10pt,leftmargin=1, rightmargin=1, innerleftmargin=15, innertopmargin=15,innerbottommargin=15, outerlinewidth=1, linecolor=light-gray]
        \begin{center}\begin{BVerbatim}[fontsize=\scriptsize]}{\end{BVerbatim}\end{center}
\end{mdframed}}

\newcounter{myCount}
\newcommand{\myCount}[1]{\refstepcounter{myCount}#1}
\newcommand{\myCountautorefname}{Example}

\newenvironment{LabeledExample}[2]{\VerbatimEnvironment\begin{mdframed}[backgroundcolor=light-gray, roundcorner=0pt, leftmargin=0, rightmargin=0, innerleftmargin=0, innerrightmargin=0,innertopmargin=5,innerbottommargin=-10, outerlinewidth=1, linecolor=light-gray]\myCount{#2}
        \begin{center}\begin{Verbatim}[samepage=true,frame=bottomline, framesep=5mm, fontsize=\tiny, label=\small{Example \arabic{myCount}: #1}, labelposition=bottomline, commandchars=\\\{\}]}{\end{Verbatim}\end{center}
\end{mdframed}}

