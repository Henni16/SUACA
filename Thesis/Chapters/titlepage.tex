\begin{titlepage}
\setlength{\topmargin}{0cm}

\begin{changetext}{2cm}{0cm}{0cm}{0cm}{0cm}
   
  

%\let\footnotesize\small \let\footnoterule\relax
\begin{center}

%\hbox{}
%\vfill
\includegraphics[width=4cm]{eule}
\vskip 1cm
Saarland University\\
Faculty of Natural Sciences and Technology I\\
Department of Computer Science\\[2ex]
\setlength{\textheight}{2cm}
\vskip 1cm
{\large \bfseries Bachelor Thesis}

\vskip .75cm
{\huge\bfseries SUACA \par}
{A tool for performance analysis of machine programs \par}

\vskip 1.5cm

submitted by\\
\vskip .25cm
{\large Hendrik Meerkamp}
\\

\vskip 1cm

submitted\\
\vskip .25cm
July 10, 2018

\vskip 1.5cm

Supervisor\\
\vskip .25cm
Prof. Dr. Sebastian Hack\\
\vskip .5cm
Advisor \\
\vskip .25cm
Andreas Abel\\
\vskip .5cm
Reviewers\\
\vskip .25cm
Prof. Dr. Sebastian Hack\\
Prof. Dr. Jan Reineke
\end{center}
\end{changetext}
\vfill
\end{titlepage}

\newpage
\thispagestyle{empty}
\mbox{}

\includepdf[pages=2]{deckblatt_BA_MA_Dipl_eid_bibo_engl.pdf}

\newpage
\thispagestyle{empty}
\mbox{}

\begin{abstract}
When trying to highly optimize your code it is essential to know how well it fits your machine. In this work we present \emph{Saarland University Architecture Code Analyzer} (\suaca), a tool that reimplements the throughput and port analysis of Intel's \iaca. Additionally it offers various options to further investigate the code's performance as well as its bottlenecks. We will discuss what its capabilities are and how the results should be understood.
\end{abstract}


\newpage
\thispagestyle{empty}
\mbox{}

\renewcommand{\abstractname}{Acknowledgments}

\begin{abstract}
    I would like to thank...
\end{abstract}

\newpage
\thispagestyle{empty}
\mbox{}